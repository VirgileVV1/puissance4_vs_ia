% Type de document : "article"
\documentclass{article}
    \addtolength{\topmargin}{-3cm}
    \addtolength{\textheight}{3cm}
    
%----- Préambule ---------

% Encodage du fichier source
\usepackage[utf8]{inputenc}

% Typographie française
\usepackage[french]{babel}

% pour les images
\usepackage{graphicx}

\usepackage{amsmath}
\usepackage{amsfonts} 
\usepackage{amssymb}

% Déclaration pour l'en-tête
\title{\textbf{Modélisation d'un puissance 4}}
\author{Adam Creusevault, Virgile Val-Villellas, Matthieu Pays}

%---- corps --------

\begin{document}

\maketitle
\thispagestyle{empty}

Un problème $Q$ est un sextuplet $(E, i, F, O, P, C)$ où :

 - $E$ est un ensemble d'états

 - $i\in E$ un état itinial

 - $F \subseteq E$ un sous-ensemble d'états finaux

 - $O = \{ o : E \nrightarrow E \} $ un ensemble d'opérateur de transformation.

 - $P = \{ p : O \times E \nrightarrow \mathbb{B} \} $ ensemble des prédicats associés définissant le cas d'applicabilité d'un opérateur sur un état.

 - $C = \{ c : O \times E \nrightarrow \mathbb{R^+} \} $ un ensemble de fonctions de coûts associées.


On définit le tableau d'un puissance 4 par une matrice $T \in \mathbb{M}_{6, 7}([\![0;2]\!])$ où $0$ représente un vide dans le tableau, $1$ représente les rouges et $2$ représente les jaunes.

$i = O_{6, 7}$ c'est-à-dire c'est une matrice remplie de $0$. Plus visuellement : 


$$
i = 
\begin{pmatrix}
0 & 0 & 0 & 0 & 0 & 0 & 0 \\
0 & 0 & 0 & 0 & 0 & 0 & 0 \\
0 & 0 & 0 & 0 & 0 & 0 & 0 \\
0 & 0 & 0 & 0 & 0 & 0 & 0 \\
0 & 0 & 0 & 0 & 0 & 0 & 0 \\
0 & 0 & 0 & 0 & 0 & 0 & 0
\end{pmatrix}$$


En ce qui concerne $F$, on a l'ensemble des matrices tel qu'ils ont 4 éléments étant égaux successivement et différents de 0. 

cond1 : $m_{i,j} = m_{i+1,j} = m_{i+2,j} = m_{i+3,j} \neq 0$

cond2 : $m_{i,j} = m_{i,j+1} = m_{i,j+2} = m_{i,j+3} \neq 0$

cond3 : $m_{i,j} = m_{i+1,j+1} = m_{i+2,j+2} = m_{i+3,j+3} \neq 0$

$F = \{ (m_{i,j})_{i\in  [\![1;6]\!], j\in [\![1;7]\!]  } |  \text{ cond1 } \vee \text{ cond2 } \vee \text{ cond3 } \}$


Pour les opérateurs de transformation, on peut prendre 

Pour $M \in \mathbb{M}_{6, 7}([\![0;2]\!])$

$\text{putPiece}(M, i, j) = M'\text{ avec }m'_{i, j} = 1\text{ ou }m'_{i, j} = 2\text{ si }m_{i, j} = 0$

$\text{putPiece2}(M, i, j) = M'\text{ avec }m'_{i-1, j} = 1\text{ ou }m'_{i-1, j} = 2\text{ si }m_{i, j} \neq 0$


$O = \{ \text{putPiece}, \text{putPiece2} \}$

Pour les prédicats de $P$, on a :

Pour $M \in \mathbb{M}_{6, 7}([\![0;2]\!])$

$\text{p1} = (\text{putPiece}, M) = \text{True si }m_{i, j} = 0\text{ False sinon} $

$\text{p2} = (\text{putPiece2}, M) = \text{True si }m_{i, j} \neq 0\text{ False sinon} $


$P = \{\text{p1}, \text{p2} \}$


Pour les coups, on a : $\forall(o,e), c(o,e) = 1$


\end{document}
%---- fin du corps
